\documentclass[11pt,a4paper]{moderncv}

\usepackage[french]{babel}   % pour dire que le texte est en francais
%% \setlength{\parindent}{0pt}
%% \usepackage{enumitem}

%% see also : https://www.latextemplates.com/template/moderncv-cv-and-cover-letter

%% http://blog.madrzejewski.com/blog/2013/01/02/creer-cv-elegant-latex-moderncv/
% ma pr{\'e}f{\'e}rence en terme de design
\moderncvtheme[black]{classic}
% d'autres valeurs possibles pour le style
%% \moderncvstyle{...}
%% classic ; casual ; oldstyle ; banking
% une autre mani{\`e}re de d{\'e}finir la couleur
%% \moderncvcolor{...}
% couleurs possibles : 'blue' (par defaut),
% 'orange', 'green', 'red', 'purple', 'grey' et 'black'

%%%%% %%%%% %%%%% %%%%% %%%%% %%%%% %%%%% %%%%% %%%%% %%%%% %%%%% %%%%% %%%%% %%%%% %%%%% 
%%%%% %%%%% %%%%% %%%%% %%%%% %%%%% %%%%% %%%%% %%%%% %%%%% %%%%% %%%%% %%%%% %%%%% %%%%% 
%%%%% %%%%% %%%%% %%%%% %%%%% %%%%% %%%%% %%%%% %%%%% %%%%% %%%%% %%%%% %%%%% %%%%% %%%%% 
%%%%% %%%%% %%%%% %%%%% %%%%% %%%%% %%%%% %%%%% %%%%% %%%%% %%%%% %%%%% %%%%% %%%%% %%%%% 
%%%%% %%%%% %%%%% %%%%% %%%%% %%%%% %%%%% %%%%% %%%%% %%%%% %%%%% %%%%% %%%%% %%%%% %%%%% 
\def\titreGeneralNewLine{General Title}
\def\titreGeneral{Titre poste CV}
\def\titreSpecialite{( Spécialité )}
\def\prenom{ Prénom }
\def\nom{ Nom }
\def\prenomNom{\prenom ~\nom }
\def\adressePhysique{ 1337 Grand Boulevard -- 61337 Section}
\def\portable{06~00~00~00~00}
\def\telephonePortable{T{\'e}l{\'e}phone Portable : \portable }
\def\eMail{ prenom.nom@fournisseur.com }
\def\pageWeb{ http://www.siteweb.com } 
\def\eMailTitle{Courriel : \eMail  }
\def\pageWebTitle{Web : \pageWeb  }
%% \def\adresseContact{%
%% 	\scriptsize{\prenomNom  -- \adressePhysique  } \\
%% 	\scriptsize{\terminalBipper  -- \eMailTitle -- \pageWebTitle} \\
%% }%
\def\adresseContact{%
	\begin{tabular}[h]{l c r}
		\includegraphics[width=0.5cm]{\logoGliderLeftt } &
		\begin{minipage}[ht]{0.90\textwidth}
			\centering
			\scriptsize{\prenomNom ~-- \adressePhysique -- %% \copyright
			\includegraphics[width=0.30cm]{\logoCreativeCommon }  \today } \\
			\scriptsize{\terminalBipper ~-- \telephonePortable} \\
			\scriptsize{\eMailTitle  ~-- \pageWebTitle} \\
		\end{minipage} &
		\includegraphics[width=0.5cm]{\logoGliderRight } \\
		%% 	&	&	\\
	\end{tabular}~\\
}%

%%%%% %%%%% %%%%% %%%%% %%%%% %%%%% %%%%% %%%%% %%%%% %%%%% %%%%% %%%%% %%%%% %%%%% %%%%% 
%%%%% %%%%% %%%%% %%%%% %%%%% %%%%% %%%%% %%%%% %%%%% %%%%% %%%%% %%%%% %%%%% %%%%% %%%%% 
%%%%% %%%%% %%%%% %%%%% %%%%% %%%%% %%%%% %%%%% %%%%% %%%%% %%%%% %%%%% %%%%% %%%%% %%%%% 
%%%%% %%%%% %%%%% %%%%% %%%%% %%%%% %%%%% %%%%% %%%%% %%%%% %%%%% %%%%% %%%%% %%%%% %%%%% 
%%%%% %%%%% %%%%% %%%%% %%%%% %%%%% %%%%% %%%%% %%%%% %%%%% %%%%% %%%%% %%%%% %%%%% %%%%% 
\usepackage{lastpage}
% \AtEndDocument{\label{LastPage}}

\def\logoGliderNorma{../img/logo_glider.png}
\def\logoGliderRight{../img/logo-glider-right.png}
\def\logoGliderLeftt{../img/logo-glider-left.png}
\def\logoCreativeCommon{../img/CreativeCommonLogo.jpeg}
\def\includeLogoGN{\includegraphics[width=0.50cm]{\logoGliderNorma }}
\def\includeLogoGR{\includegraphics[width=0.50cm]{\logoGliderRight }}
\def\includeLogoGL{\includegraphics[width=0.50cm]{\logoGliderLeftt }}
\def\includeLogoCC{\includegraphics[width=0.25cm]{\logoCreativeCommon }}

\pagestyle{fancy}
\def\makestylefancyContent{%
	\fancyhf{}
	\fancyhead[LE]{
		\includegraphics[width=0.5cm]{\logoGliderLeftt }
		\hfill
		\prenomNom 
		\hfill
		\titreGeneral  -- \titreSpecialite
	}
	\fancyfoot[LE]{
		\includegraphics[width=0.5cm]{\logoGliderLeftt } \hfill
		\includeLogoCC  \prenomNom  -- \today \hfill %% \copyright
		\thepage /\pageref{LastPage}
	}
	
	\fancyhead[RO]{
		\titreGeneral  -- \titreSpecialite
		\hfill
		\prenomNom 
		\hfill
		\includegraphics[width=0.5cm]{\logoGliderRight }
	}
	\fancyfoot[RO]{
		\thepage /\pageref{LastPage} \hfill
		\includeLogoCC  \prenomNom  -- \today \hfill %% \copyright
		\includegraphics[width=0.5cm]{\logoGliderRight }
	}
	\renewcommand{\headrulewidth}{0.25pt}
	\renewcommand{\footrulewidth}{0.5pt}
}%
\makestylefancyContent

\def\motsClefs{LaTeX;PDF;hyperlinks;bioinformatique;bio-informatique;bases de donn{\'e}es;SQL;%
				JAVA;J2EE;Swing;C / C++;Perl;Python;PHP;HTML;CSS;JavaScript;%
				MySQL;PostgreSQL;ORACLE;data mining / analyse de donn{\'e}es;%
				UML;algorithmique;techniques de biologie; bio-informatique;%
				bash;shell;Git;Maven;Mockito;MockNeat;BioJava;BioPerl;BioPython%
				GNU / Linux (Ubuntu);UNIX;Mac OS X;Windows;%
				UML;MERISE;G{\'e}nie Logiciel;Tests Unitaires;JUnit%
				Biochimie;G{\'e}n{\'e}tique;Biologie Cellulaire;%
				Scrum;eXtreme Programming;M{\'e}thodes Agiles;%
				Pipeline Pilot;Accelrys;Component;Protocol;%
				RDF;Resource Description Framework;RML;RDF Mapping Language;Apache Jena;Semantic Web%
}

% \usepackage{hyperref}
% \hypersetup
% {
	% pdfauthor={\prenomNom}, 
	% pdfsubject={\titreGeneral ~-- \titreSpecialite},
	% pdftitle={\titreGeneral ~-- \titreSpecialite},
	% pdfkeywords={\motsClefs},
	% pdfcreator={PDFLaTeX (creation)}, 
	% pdfproducer={PDFLaTeX (production)}
% }
%% http://www.andy-roberts.net/writing/latex/pdfs -- for some properties...

% \usepackage[pdftex,
			% pdfauthor={\prenomNom},
			% pdftitle={\titreGeneral ~-- \titreSpecialite},
			% pdfsubject={\titreGeneral ~-- \titreSpecialite},
			% pdfkeywords={\motsClefs},
			% pdfproducer={PDFLaTeX (creation)},
			% pdfcreator={PDFLaTeX (production)}]{hyperref}
%% https://tex.stackexchange.com/questions/26529/how-can-i-generate-pdf-metadata-from-latex

\AfterPreamble{\hypersetup{
			pdfauthor={\prenomNom},
			pdftitle={\titreGeneral ~-- \titreSpecialite},
			pdfsubject={\titreGeneral ~-- \titreSpecialite},
			pdfkeywords={\motsClefs },
			pdfproducer={PDFLaTeX (creation)},
			pdfcreator={PDFLaTeX (production)}
			%% urlcolor=blue,
}}
%% https://tex.stackexchange.com/questions/79395/option-clash-for-hyperref-package

%%%%% %%%%% %%%%% %%%%% %%%%% %%%%% %%%%% %%%%% %%%%% %%%%% %%%%% %%%%% %%%%% %%%%% %%%%% 
%%%%% %%%%% %%%%% %%%%% %%%%% %%%%% %%%%% %%%%% %%%%% %%%%% %%%%% %%%%% %%%%% %%%%% %%%%% 
%%%%% %%%%% %%%%% %%%%% %%%%% %%%%% %%%%% %%%%% %%%%% %%%%% %%%%% %%%%% %%%%% %%%%% %%%%% 
%%%%% %%%%% %%%%% %%%%% %%%%% %%%%% %%%%% %%%%% %%%%% %%%%% %%%%% %%%%% %%%%% %%%%% %%%%% 
%%%%% %%%%% %%%%% %%%%% %%%%% %%%%% %%%%% %%%%% %%%%% %%%%% %%%%% %%%%% %%%%% %%%%% %%%%% 

%% \usepackage[top=1.1cm, bottom=1.1cm, left=2cm, right=2cm]{geometry}
% gauche, haut, droite, bas, entete, ente2txt, pied, txt2pied
\usepackage{vmargin}
\setmarginsrb{1.0cm}{0.2cm}{1.0cm}{0.50cm}{15pt}{10pt}{15pt}{45pt}

% Largeur de la colonne pour les dates
\setlength{\hintscolumnwidth}{2.25cm} %% 2.50cm

%% Une ent{\^e}te classique
\firstname{\prenom }			%% \firstname{\prenom }
\familyname{\nom }				%% \familyname{\nom }
\title{\titreGeneral }			%% \title{\titreGeneral }
%% \title{\titreGeneral \newline \titreSpecialite} %% titreGeneralNewLine
\address{\adressePhysique }		%% \address{\adressePhysique }
\email{\eMail }					%% \email{\eMail }
\homepage{\pageWeb }			%% \homepage{\pageWeb }
\mobile{\portable }				%% \mobile{\portable }
\quote{ QuotationCitation }		%% \quote{ QuotationCitation }

% d'autres valeurs possibles
%% \phone{00 00 00 00 00}
%% \fax{00 00 00 00 00}
%% \photo[64pt][0.4pt]{../img/logo_glider.png}

%% http://lataix-sebastien.developpez.com/tutoriels/latex/tutoriel-moderncv/
% les donn{\'e}es suivantes sont aussi optionnelles donc {\`a} commenter si on n'en veut pas
%% \phone[mobile]{\portable } %% \phone[mobile]{06~12~34~56~78}
%% \phone[fixed]{01~01~01~01~01}
%% \phone[fax]{01~01~01~01~01}
\social[linkedin]{ LinkedInProfile }		%% https://www.linkedin.com/in/.../
\social[github]{ GitHubProfile }			%% https://github.com/...
%% \social[facebook]{ FaceBookProfile }		%% https://www.facebook.com/...
\social[twitter]{ TwitterPage }				%% https://twitter.com/...
%% \extrainfo{informations compl{\'e}mentaires.}
%% \quote{ \Quotation } 					%% toujours optionnel, se place avant le corps du CV

\begin{document}

	%% \makestylefancyContent
	\maketitle

\section{Informations}
	Introduction Text !!~\\

\section{Exp{\'e}rience professionnelle}
	%% \cventry{years}{degree/job title}{institution/employer}{localization}{grade}{description}
	%% \cventry{DATUM}{TITRE}{ENTREPRISE}{CONTRAT}%
	%% 		{\newline INTITULE++}{%
	%% 	\begin{itemize}
	%% 		\item[$\rightarrow$] ELEMENTUN
	%% 		\item[$\bullet$] ELEMENTDEUXETPLUS
	%% 	\end{itemize}}

	\cventry{DATUM}{TITRE}{ENTREPRISE}{CONTRAT}%
			{\newline INTITULE++}{%
		\begin{itemize}
			\item[$\rightarrow$] ELEMENTUN
			\item[$\bullet$] ELEMENTDEUXETPLUS
		\end{itemize}}

\section{Formation}
	%% on peut mettre ici de 3 {\`a} 6 arguments qui peuvent {\^e}tre laiss{\'e}s vides
	\cventry{Year}{Diploma}{\newline School}{Location}	{}{}{}

\section{Licences et Certifications}
	\cventry{Year}{Diploma}{\newline School}{Location}	{}{}{}

\section{Expériences de bénévolat}
	\cventry{years}{degree/job title}{institution/employer}{localization}{grade}{description}

\section{Compétences}
	\cvdoubleitem{ Item1 }{ Description1 }{ Item2 }{ Description2 }

\section{Recommandations}
	\cvcomputer{ Item1 }{ Description1 }{ Item2 }{ Description2 }

\section{Réalisations}
	\cvitem{Projets}{ Sur GitHub (par exemple) }
	\cvitem{Langues}{ Anglais... }
	\cvitem{Organisations}{ associations... }
	\cvitem{Publications}{ citations, references... }
	\cvitem{Lectures}{ READINGS }
	\cvcomputer{ Item1 }{ Description1 }{ Item2 }{ Description2 }

\section{Centres d'int{\'e}r{\^e}ts}
	\cvitem{ReadingS}{ READINGS }
	%% \cvitemwithcomment{Jeux de Soci{\'e}t{\'e}}{Jeux de r{\^o}les, jeux de plateaux. }{Joueur et MJ / Animateur}
		%% \cvlistdoubleitem{L'Appel De Cthulhu}		{CyberPunk et assimil{\'e}s}
		%% \cvlistdoubleitem{Sombre / Sombre Z{\'e}ro}	{StarWars : Force et Destin{\'e}e}
		%% \cvlistdoubleitem{NumenEra}					{Vampire : La Mascarade}
	%% \cvitemwithcomment{Element}{Detail}{Commentaire}
	\cventry{Year}{WHAT}{CONTENT}{Location}	{MORE1}{MORE2}{MORE3}

%% ~\\

%% \section{divers}
%% % plein de choses diff{\'e}rentes pour pr{\'e}senter vos comp{\'e}tences
%% \cvitemwithcomment{Comp{\'e}tence}{Niveau}{Commentaire}
%% \cvdoubleitem{Categorie 1}{XXX, YYY, ZZZ}{Categorie 2}{XXX, YYY, ZZZ}
%% \cvitem{Quelque chose}{Description}
%% % une liste
%% \cvlistitem{Item 1}
%% \cvlistitem{Item 2}
%% \cvlistitem{Item 3}
%% % une double liste
%% \cvlistdoubleitem{Item 1}{Item 3}
%% \cvlistdoubleitem{Item 2}{Item 4}

%% \section{Une section}
% pour votre cursus scolaire
%% \cventry{ann{\'e}e--ann{\'e}e}{Diplome}{Institution}{Ville}{\textit{Classe}}{Description}
% les {} 3 {\`a} 6 sont optionnelles
% exemple
%% \cventry{2012 -- 2013}{Licence Professionnelle}{IUT Clermont-Ferrand (63)}{}{}{Sp{\'e}cialit{\'e} Administration et S{\'e}curit{\'e} des R{\'e}seaux}
% pour une exp{\'e}rience pro
%% \cventry{ann{\'e}e--ann{\'e}e}{Titre de l'emploi}{Employeur}{Ville}{}{Description de 1 ou 2 lignes.\newline{}%
% exemple avec une liste (un peu plus complexe)
%% \cventry{Avril 2012\\{\`a} Octobre 2012}{Administrateur syst{\`e}mes}{Vesalis}{Clermont-Ferrand}{France}{
%% \begin{itemize}%
%% \item Mise en place d'une solution de r{\'e}partition de charge sur un serveur IBM Blade Center (~6 serveurs),
%% \item Proposition d'une architecture de haute disponibilit{\'e} (HA),
%% \item Changement d'h{\'e}bergeur, r{\'e}installation de toutes les applications et services sur un serveur Windows Server 2008 R2 (configuration d'un SVN, serveur web, base de donn{\'e}es, FTP),
%% \item Adaptation du code des applications de la soci{\'e}t{\'e} pour le rendre compatible Linux.\newline{}
%% \end{itemize}}%

\end{document}
